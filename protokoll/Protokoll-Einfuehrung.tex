%%%%%%%%%%%%%%%%%%%%%%%%%%%%%%%%%%%%%%%%%%%%%%%%%%%%%%%%%%%%%%%%%%%%%%%%%%%%%%%%
\chapter{Einführung}
%%%%%%%%%%%%%%%%%%%%%%%%%%%%%%%%%%%%%%%%%%%%%%%%%%%%%%%%%%%%%%%%%%%%%%%%%%%%%%%%

Im Rahmen dieser Belegarbeit soll ein Ansatz entwickelt werden, der es ermöglicht mit Java oder Scala auf heterogenen Clustersystemen mit Grafikkarten zu rechnen. Es wurde sich für eine Kombination aus Spark für die Kommunikation im Cluster und Rootbeer für die Grafikkartenprogrammierung entschieden.

Spark vereinfacht die ausfallsichere Programmierung von Clustern mittels des Map-Reduce-Programmiermodells und stellt zahlreiche Bibliotheken z.B. für Graphenalgorithmen und statistische Analysen zur Verfügung.

Für Rootbeer wurde sich entschieden, weil es das Schreiben von CUDA-Kernels aus Java heraus erlaubt. Die so geschriebenen Kernel-Funktionen können dann sowohl auf Grafikkarten als auch auf dem Host ausgeführt werden.

Zuerst werden in den Kapiteln \ref{sct:spark} und \ref{sct:rootbeer} die benutzten Frameworks genauer vorgestellt, in \autoref{sct:implementation} wird auf die eigene Implementierung eingegangen und in Kapitel~\ref{sct:benchmarks} werden Benchmarks, die mit dieser Implementierung angefertigt wurden, ausgewertet.
