
%%%%%%%%%%%%%%%%%%%%%%%%%%%%%%%%%%%%%%%%%%%%%%%%%%%%%%%%%%%%%%%%%%%%%%%%%%%%%%%%
\chapter{Rootbeer}
\label{sct:rootbeer}
%%%%%%%%%%%%%%%%%%%%%%%%%%%%%%%%%%%%%%%%%%%%%%%%%%%%%%%%%%%%%%%%%%%%%%%%%%%%%%%%

Rootbeer\cite{pratt2012rootbeer} ist ein von Philip C. Pratt-Szeliga entwickeltes Programm welches das Schreiben von CUDA-Kerneln in Java ermöglicht.
Zum aktuellen Zeitpunkt, August 2016, hat Rootbeer leider noch Beta-Status und wurde seit ca. einem Jahr nicht weiterentwickelt\cite{rootbeergithub}.

Der Rootbeerquellcode enthält Gerüste, um den Rootbeerkernel nicht nur nach CUDA, sondern auch nach OpenCL zu übersetzen, aber diese Funktionalität ist noch nicht fertiggestellt.

Andere Ansätze für Grafikkartenprogrammierung innerhalb von Java bzw. Scala stellen simple, teilweise direkt von den Headern der OpenCL-Spezifikation kompilierte, oder aber auch komplexere objektorientierte Java-Schnittstellen für OpenCL und CUDA zur Verfügung, so z.B. JogAmp JOCL\cite{jogampcl}, JOCL\cite{jocl}, JavaCL\cite{javacl} und jCUDA\cite{jcuda} bzw. ScalaCL\cite{scalacl} und Firepile\cite{firepile} für Scala.
Diese erfordern jedoch die Kenntnis von OpenCL bzw. CUDA und der Nutzer muss sich selbstständig um Serialisierung und Host-GPU-Transfers kümmern.
ScalaCL und Firepile sind außerdem auch noch in der Entwicklung.

Desweiteren gibt es fertige Bibliotheken, die auf Grafikkarten portierte Funktionen zur Verfügung stellen, welche jedoch nur eine beschränkte Einsetzbarkeit haben.

Eine Alternative zu Rootbeer stellt das ursprünglich von AMD entwickelte Aparapi\cite{aparapi}, welches seit 2011 Open Source ist, dar.
Wie in Rootbeer ist es möglich in Java Kernel zu schreiben, die von Aparapi in OpenCL übersetzt werden.
Jedoch werden nur eindimensionale Felder unterstützt, im Gegensatz zu Rootbeer welches auch komplexe Objekte serialisieren kann.\\


%%%%%%%%%%%%%%%%%%%%%%%%%%%%%%%%%%%%%%%%%%%%%%%%%%%%%%%%%%%%%%%%%%%%%%%%%%%%%%%%
%\section{Funktionsweise}
%%%%%%%%%%%%%%%%%%%%%%%%%%%%%%%%%%%%%%%%%%%%%%%%%%%%%%%%%%%%%%%%%%%%%%%%%%%%%%%%


%%%%%%%%%%%%%%%%%%%%%%%%%%%%%%%%%%%%%%%%%%%%%%%%%%%%%%%%%%%%%%%%%%%%%%%%%%%%%%%%
%\subsection{Nutzung}
%%%%%%%%%%%%%%%%%%%%%%%%%%%%%%%%%%%%%%%%%%%%%%%%%%%%%%%%%%%%%%%%%%%%%%%%%%%%%%%%

Für die Nutzung von Rootbeer ist das Java SE Development Kit 7 und das NVIDIA CUDA Toolkit notwendig.
Rootbeer funktioniert noch nicht mit JDK 8 und unterstützt offiziell nur JDK 6 \cite{rootbeerjdk6}, aber in den hier durchgeführten Beispielen gab es keine Probleme mit JDK 7.

Zuerst muss der Nutzer das \lstinline!org.trifort.rootbeer.runtime.Kernel!-Interface implementieren und die kompilierten Klassen als JAR-Archive nochmals an den Rootbeer-Compiler geben, siehe \autoref{sct:kernelimplementation} und \autoref{sct:compilation}.
Es ist zu beachten, dass die an Rootbeer übergebene Datei auf \texttt{.jar} enden muss, insbesondere führen Dateinamen wie \texttt{gpu.jar.tmp} zu einer Fehlermeldung.\\

Der Rootbeer-Compiler nutzt dann Soot\cite{sootsite,sootretrospective}, um den Java Bytecode der Kernel-Implementation in Jimple zu übersetzen.
Jimple ist eine vereinfachte Zwischendarstellung von Java-Bytecode in Drei-Address-Code mit nur 15 verschiedenen Befehlen. Java-Bytecode hingegen hat ca. 200 verschiedene Befehle.
Der Jimple-Code wird dann analysiert und in CUDA-Quellcode übersetzt, welcher dann mit dem NVIDIA-Compiler kompiliert wird.
All das geschieht automatisch, aber die Zwischenschritte kann man zur Fehlersuche unter Linux in \lstinline!$HOME/.rootbeer/! einsehen.
Die so erstellte cubin-Datei wird zusammen mit \texttt{Rootbeer.jar} dem JAR-Archiv des Nutzers hinzugefügt.

Die zweite große Vereinfachung, die Rootbeer zur Verfügung stellt, ist die Automatisierung des Datentransfers zwischen GPU und CPU.
Das besondere hierbei ist, dass Rootbeer die Nutzung von beliebigen, also insbesondere auch nicht-primitiven Datentypen erlaubt.
Diese Datentypen serialisiert Rootbeer automatisch und unter Nutzung aller CPU-Kerne und transferiert sie danach auf die Grafikkarte.

Diese zwei Vereinfachungen obig machen die erste Nutzung von Rootbeer verglichen zu anderen Lösungen sehr einfach, sodass Rootbeer insbesondere für das Erstellen von Prototypen günstig ist.
In Kontrast dazu ist es jedoch auch möglich sehr nah an der Grafikkarte zu programmieren.
Dafür kann man mit Rootbeer manuell die Kernel-Konfiguration angeben, mehrere GPUs ansprechen, shared memory nutzen und auch über die \lstinline!RootbeerGpu!-Klasse CUDA-Befehle wie \lstinline!syncthreads! benutzen.
